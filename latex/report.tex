%%%%%%%%%%%%%%%%%%%%%%%%%%%%%%%%%%%%%%%%%
% Simple Sectioned Essay Template
% LaTeX Template
%
% This template has been downloaded from:
% http://www.latextemplates.com
%
% Note:
% The \lipsum[#] commands throughout this template generate dummy text
% to fill the template out. These commands should all be removed when 
% writing essay content.
%
%%%%%%%%%%%%%%%%%%%%%%%%%%%%%%%%%%%%%%%%%

%----------------------------------------------------------------------------------------
%	PACKAGES AND OTHER DOCUMENT CONFIGURATIONS
%----------------------------------------------------------------------------------------

\documentclass[12pt]{article} % Default font size is 12pt, it can be changed here

\usepackage{geometry} % Required to change the page size to A4
\geometry{a4paper} % Set the page size to be A4 as opposed to the default US Letter
\usepackage{cite}
\usepackage{graphicx} % Required for including pictures
\usepackage{mathrsfs}
\usepackage{amsbsy}
\usepackage{float} % Allows putting an [H] in \begin{figure} to specify the exact location of the figure
\usepackage{wrapfig} % Allows in-line images such as the example fish picture
\usepackage{amsmath}
\usepackage{lipsum} % Used for inserting dummy 'Lorem ipsum' text into the template
\usepackage{subcaption}
\usepackage[utf8]{inputenc}
\usepackage{bm}
\usepackage{amsmath,amsfonts,amsthm}
\usepackage{mathtools}
\usepackage{indentfirst}
\usepackage{tcolorbox}
\newcommand{\R}{\mathbb{R}}
\newcommand{\N}{\mathbb{N}}
\setlength{\topmargin}{-.3in} 
\setlength{\textheight}{8.4in}
\setlength{\textwidth}{5.9in}      
\setlength{\oddsidemargin}{0.2in}  
\setlength{\evensidemargin}{0.2in} 
\usepackage{pifont}

\newcommand{\vertiii}[1]{{\left\vert\kern-0.25ex\left\vert\kern-0.25ex\left\vert #1 
    \right\vert\kern-0.25ex\right\vert\kern-0.25ex\right\vert}}
\renewcommand{\div}{\nabla \cdot}
\newcommand{\bsigma}{\mbox{\boldmath${\sigma}$}}
\newcommand{\bSigma}{\mbox{\boldmath${\Sigma}$}}
\newcommand{\tbsigma}{\mbox{\boldmath${\widetilde{\sigma}}$}}
\newcommand{\tb}{\mbox{\boldmath${\widetilde{b}}$}}
\newcommand{\btheta}{\mbox{\boldmath${\theta}$}}
\newcommand{\bdelta}{\mbox{\boldmath${\delta}$}}
\newcommand{\btau}{\mbox{\boldmath${\tau}$}}
\newcommand{\bpsi}{\mbox{\boldmath${\psi}$}}
\newcommand{\bphi}{\mbox{\boldmath${\phi}$}}
\newcommand{\Beta}{\mbox{\boldmath${\eta}$}}
\newcommand{\bu}{\bm{u}}
\newcommand{\br}{\bm{r}}
%\setlength\parindent{0pt} % Uncomment to remove all indentation from paragraphs


\begin{document}

\title{Navier-Stokes Generative Adversarial Network}
\date{}
\maketitle
\section{Mathematical model}
We consider residual of Navier-Stokes equation that
\begin{equation}
    r_{cty} = \nabla \cdot \bu
\end{equation}
\begin{equation}
    r_{u} = \bu_t + \sigma \bu + (\bu \cdot \nabla)\bu  - \mu\Delta\bu + \nabla p
\end{equation}
% \begin{equation}
%     r_{c} = c_t + \sigma c + (\bu \cdot \nabla)c - \mu\Delta c
% \end{equation}

Denote the velocity as $\bu = [u_1, u_2]^T$ and pressure $p$.
In discrete form, we have the residuals $\br = [r_{cty}, r_{u_1}, r_{u_2}]$ at point $(x_i,y_j,t_n)$ that
\begin{eqnarray}\label{res_cty}
    r_{cty}(x_i,y_j,t_n) &=& \frac{u_1(x_{i+1},y_{j},t_{n})-u_1(x_{i-1},y_{j},t_{n})}{2dx}\nonumber\\
                         & &+ \frac{u_2(x_{i},y_{j+1},t_{n})-u_2(x_{i},y_{j-1},t_{n})}{2dy}
\end{eqnarray}
\begin{eqnarray}\label{res_u1}
    r_{u_1}(x_i,y_j,t_n) &=& \rho\frac{u_1(x_i,y_j,t_{n+1}) - u_1(x_i,y_j,t_{n})}{dt} + \sigma u_1(x_i,y_j,t_{n})\nonumber\\
                         & & + \rho u_1\frac{u_1(x_{i+1},y_{j},t_{n})-u_1(x_{i-1},y_{j},t_{n})}{2dx}\nonumber\\
                         & & + \rho u_2\frac{u_1(x_{i},y_{j+1},t_{n})-u_1(x_{i},y_{j-1},t_{n})}{2dy}\nonumber\\
                         & & + \mu \frac{-u_1(x_{i+1},y_{j},t_{n})+2u_1(x_{i},y_{j},t_{n})-u_1(x_{i-1},y_{j},t_{n})}{dx^2}\nonumber\\
                         & & + \mu \frac{-u_1(x_{i},y_{j+1},t_{n})+2u_1(x_{i},y_{j},t_{n})-u_1(x_{i},y_{j-1},t_{n})}{dy^2}\nonumber\\
                         & & + \frac{p(x_{i+1},y_{j},t_{n}) - p(x_{i-1},y_{j},t_{n})}{2dx}
\end{eqnarray}
\begin{eqnarray}\label{res_u2}
    r_{u_2}(x_i,y_j,t_n) &=& \rho\frac{u_2(x_i,y_j,t_{n+1}) - u_2(x_i,y_j,t_{n})}{dt} + \sigma u_2(x_i,y_j,t_{n})\nonumber\\
                         & & + \rho u_1\frac{u_2(x_{i+1},y_{j},t_{n})-u_2(x_{i-1},y_{j},t_{n})}{2dx}\nonumber\\
                         & & + \rho u_2\frac{u_2(x_{i},y_{j+1},t_{n})-u_2(x_{i},y_{j-1},t_{n})}{2dy}\nonumber\\
                         & & + \mu \frac{-u_2(x_{i+1},y_{j},t_{n})+2u_2(x_{i},y_{j},t_{n})-u_2(x_{i-1},y_{j},t_{n})}{dx^2}\nonumber\\
                         & & + \mu \frac{-u_2(x_{i},y_{j+1},t_{n})+2u_2(x_{i},y_{j},t_{n})-u_2(x_{i},y_{j-1},t_{n})}{dy^2}\nonumber\\
                         & & + \frac{p(x_{i},y_{j+1},t_{n}) - p(x_{i},y_{j-1},t_{n})}{2dy}
\end{eqnarray}

% \begin{equation*}\label{res_cty}
%     r_{cty}(x_i,y_j,t_n) = \frac{u_1(x_{i+1},y_{j},t_{n})-u_1(x_{i-1},y_{j},t_{n})}{2dx}+ \frac{u_2(x_{i},y_{j+1},t_{n})-u_2(x_{i},y_{j-1},t_{n})}{2dy}
% \end{equation*}
\section{MLP}
\subsection{Proper orthogonal decomposition}
Since the cost to train the network with the data of velocity $\bu$ and pressure $p$ directly is expensive, we apply the proper orthogonal decomposition (POD) to represent $\bu$ and $p$  on grid. And with the help of the basis functions, we can denote $\bu$ and $p$ as a linear combination that
\begin{equation}
    \bu(t_n) \approx \sum_{i=1}^{N_1} c1_i(t_n) \bm{\phi}_i, p\approx \sum_{i=1}^{N_2} c2_i(t_n) \psi_i
\end{equation}
where $c1$ and $c2$ are the coefficients of velocity and pressure.
\subsection{Network Structure}
Then, we set the network $\mathcal{N}(c)$ as (blank).
\section{Physics loss functions}
Here, the network is given the coefficients at time $t_n$ and predicts the coefficients at time $t_{n+1}$. Consider a pair of coefficents for velocity and pressure $[\bm{c}1_n, \bm{c}2_n]$ at time $t_n$.
Then, we have the predict labels as $[\bm{\hat{c}}1_{n+1}, \bm{\hat{c}}2_{n+1}]$. By the basis functions $\bm{\phi}(t_{n+1})$ and $\bm{\psi}(t_{n+1})$ from POD, we can recover the predict velocity  $\hat{\bm{u}}(t_{n+1})$ and pressure $\hat{p}(t_{n+1})$. Based on the equations (\ref{res_cty}, \ref{res_u1}, \ref{res_u2}), we can obtain the residuals $\hat{\br}$ of $\hat{\bm{u}}(t_{n+1})$ and $\hat{p}(t_{n+1})$ accordingly.






\bibliographystyle{plain}
\bibliography{list}
\end{document}